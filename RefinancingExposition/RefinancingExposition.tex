\documentclass{article}
\usepackage[utf8]{inputenc}
\usepackage{graphicx}
\title{Mortgage Refinancing}
\author{Matthew Zhang}
\date{May 2018}

\begin{document}

\maketitle

\section{Exposition}

\quad When refinancing from a shorter term loan with a lower overall payment but a higher monthly payment to a longer term loan with a higher overall payment but a lower monthly payment, one may deem it worthwhile to refinance to the longer term loan for a multitude of reasons. If one is short on money in the moment and cannot pay off the larger monthly payments, the shorter term loan may be more manageable despite the higher overall cost. However, if one has the monetary capacity to invest some money along with the mortgage payment, they may find that refinancing may save them money overall despite the extra overall cost. 

In our case we are considering refinancing from a 15 year mortgage with an interest rate of 2.5 \% to a mortgage 30 years long with an interest rate of 4.8 \%. The monthly payment for the 30 year loan is lower than the one that is 15 years long, and so switching to the 30 year mortgage would free up a certain amount of money every month. This money can be invested by the customer and used to generate money. If the return rate of the investment is high enough, the amount invested can exceed the difference in cost between the two mortgages, making a refinance to the 30 year loan with a seemingly higher overall cost worthwhile in the long term.

From the data presented in the spreadsheet it is determined that the break even point is at a return rate of 5.1 \% annually. However, even if the interest rate is over 5.1 \%, it may still not be worth refinancing since you are essentially paying twice the cost of the 15  year loan until the 30 year term is up to reap the benefits. Whether or not the customer should refinance depends on the return rate they will get from the investment and their current financial situation.


\end{document}
