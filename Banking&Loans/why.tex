\documentclass{article}
\usepackage[utf8]{inputenc}
\usepackage{graphicx}
\title{Some Applied Mathematics: Series, Banking, and Loans}
\author{Matthew Zhang}
\date{March 2018}

\begin{document}

\maketitle


\section*{Interest Bearing Accounts}
(a)\newline
$S_0=10000\newline
S_1= S_0(1.05)\newline
S_2 = S_1(1.05)\newline
S_3 = S_2(1.05)\newline
S_4 = S_3(1.05)\newline
S_5 = S_4(1.05)\newline 
S_n = S_{n-1}(1.05)\newline$
\linebreak 
\linebreak 
(b)\newline
$
S_n = S_{n-1}(1.05)\newline
S_n = 10000(1.05)^n\newline
S_n = 10000(1+ 0.05)^n\newline
$
Expand $(1+ 0.05)^n$ using Binomial Theorem \newline
$$
S_n = 10000\sum_{k=0}^{n} {{n \choose k} 1^{k-n}+0.05^n}\newline
$$
Remove $1^{k-n}$ term\newline
$$
S_n = 10000\sum_{k=0}^{n} {{n \choose k} 0.05^n}\newline
$$
(c)\newline
From part (a)
$S_n = S_{n-1}(1.05)\newline$
\newline
Plugging in $S_{n-1}\newline$
$S_n = S_{n-2}(1.05)(1.05) \newline
S_n = S_{n-2}(1.05)^2 \newline
$
\newline
Continuing yields \newline
$S_n = S_{n-3}(1.05)^3 \newline
S_n = S_{n-4}(1.05)^4 \newline
S_n = S_{n-n}(1.05)^n \newline
S_n = S_{0}(1.05)^n \newline
S_n = 10000(1.05)^n \newline
$

\section*{Loans}

\section*{Fractional Reserve Banking}

\end{document}
