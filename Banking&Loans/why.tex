\documentclass{article}
\usepackage[utf8]{inputenc}
\usepackage{graphicx}
\title{Some Applied Mathematics: Series, Banking, and Loans}
\author{Matthew Zhang}
\date{March 2018}

\begin{document}

\maketitle


\section*{Interest Bearing Accounts}
(a)\newline
$S_0=10000\newline
S_1= S_0(1.05)\newline
S_2 = S_1(1.05)\newline
S_3 = S_2(1.05)\newline
S_4 = S_3(1.05)\newline
S_5 = S_4(1.05)\newline 
S_n = S_{n-1}(1.05)\newline$
\linebreak 
\linebreak 
(b)\newline
$
S_n = S_{n-1}(1.05)\newline
S_n = 10000(1.05)^n\newline
S_n = 10000(1+ 0.05)^n\newline
$
Expand $(1+ 0.05)^n$ using Binomial Theorem \newline
$$
S_n = 10000\sum_{k=0}^{n} {{n \choose k} 1^{k-n}+0.05^n}\newline
$$
Remove $1^{k-n}$ term\newline
$$
S_n = 10000\sum_{k=0}^{n} {{n \choose k} 0.05^n}\newline
$$
(c)\newline
From part (a)
$S_n = S_{n-1}(1.05)\newline$
\newline
Plugging in $S_{n-1}\newline$
$S_n = S_{n-2}(1.05)(1.05) \newline
S_n = S_{n-2}(1.05)^2 \newline
$
\newline
Continuing yields \newline
$S_n = S_{n-3}(1.05)^3 \newline
S_n = S_{n-4}(1.05)^4 \newline
S_n = S_{n-n}(1.05)^n \newline
S_n = S_{0}(1.05)^n \newline
S_n = 10000(1.05)^n \newline
$

\section*{Loans}
To determine a series we examine a case where the loan is paid off in n payments\newline
The amount left after first payment is given below
$$L_0(1+\frac{r}{12})-m$$
Where m represents the monthly payment\newline
For the second payment, we multiply what is left by the monthly interest rate and subtract the monthly payment m again
$$(L_0(1+\frac{r}{12})-m)(1+\frac{r}{12})-m $$
$$L_0(1+\frac{r}{12})^2-m(1+\frac{r}{12})-m $$
Doing the same for the third payment yields
$$(L_0(1+\frac{r}{12})^2-m(1+\frac{r}{12})-m)(1+\frac{r}{12})-m $$
$$L_0(1+\frac{r}{12})^3-m(1+\frac{r}{12})^2-m(1+\frac{r}{12})-m $$
By examining the expressions one notices an emerging pattern
$$L_0(1+\frac{r}{12})-m$$
$$L_0(1+\frac{r}{12})^2-m(1+\frac{r}{12})-m $$
$$L_0(1+\frac{r}{12})^3-m(1+\frac{r}{12})^2-m(1+\frac{r}{12})-m $$
Therefore we can write a summation to represent this pattern and extend the expression to the nth payment
$$L_0(1+\frac{r}{12})^n- \sum_{k=0}^{n-1} m(1+\frac{r}{12})^k $$
The series represented here is geometric with common ratio $(1+\frac{r}{12})$ so we can write an expression for the sum
$$L_0(1+\frac{r}{12})^n- \sum_{k=0}^{n-1} m(1+\frac{r}{12})^k $$
$$L_0(1+\frac{r}{12})^n- \frac{m(1-(1+\frac{r}{12})^{n})}{1-(1+\frac{r}{12})}$$
$$L_0(1+\frac{r}{12})^n- \frac{m(1-(1+\frac{r}{12})^{n})}{\frac{r}{12}}$$
After n payments, it should be fully paid off so we can say that
$$0=L_0(1+\frac{r}{12})^n- \frac{m(1-(1+\frac{r}{12})^{n})}{\frac{r}{12}}$$
$$\frac{m(1-(1+\frac{r}{12})^{n})}{\frac{r}{12}}=L_0(1+\frac{r}{12})^n$$
$$m=\frac{\frac{r}{12}L_0(1+\frac{r}{12})^n}{{1-(1+\frac{r}{12})^{n}}}$$

\section*{Fractional Reserve Banking}

\end{document}
